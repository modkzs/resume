% !TEX TS-program = xelatex
% !TEX encoding = UTF-8 Unicode
% !Mode:: "TeX:UTF-8"

\documentclass{resume}
\usepackage{zh_CN-Adobefonts_external} % Simplified Chinese Support using external fonts (./fonts/zh_CN-Adobe/)
%\usepackage{zh_CN-Adobefonts_internal} % Simplified Chinese Support using system fonts
\usepackage{linespacing_fix} % disable extra space before next section
\usepackage{cite}

\begin{document}
\pagenumbering{gobble} % suppress displaying page number

\name{何逸轩}
% \basicInfo{hyx\_ict@163.com}{(+86) 185-1023-8192}{北京市海淀区中国科学院南路6号}


\basicInfo{
  \email{hyx\_ict@163.com} \textperiodcentered\
  \phone{(+86) 185-1023-8192} \textperiodcentered\
  \linkedin[何逸轩]{https://www.linkedin.com/in/逸轩-何-b9156bbb/}}

\section{\faGraduationCap\  教育背景}
\datedsubsection{\textbf{中科院计算所(保研)}, 硕士}{2015.9 -- 2018.6}
\datedsubsection{\textbf{北京邮电大学学}, 学士, 排名:9/100}{2011.9 -- 2015.6}

\section{\faUsers\ 项目经历}
\datedsubsection{\textbf{TensorFlow on Spark}}{2016年4月 -- 2016年12月}
\role{Java, C++, Scala}{实验室项目}
\begin{onehalfspacing}
将 TensorFlow 移植到 Spark 上。整体使用 java 开发,同时封装了部分 Tensorflow 的 C++ 接口。
\begin{itemize}
  \item 支持同步、异步、单机训练、多机训练等多种训练方式。
  \item 对 TensorFlow 进行了高层封装,支持 Layer 等高层接口
\end{itemize}
\end{onehalfspacing}

\datedsubsection{\textbf{分布式深度学习框架}}{2016年6月 -- 至今}
\role{C++, Golang}{个人项目}
\begin{onehalfspacing}
自己实现的分布式深度学习框架,可以自配置神经网络,支持自动求导
\begin{itemize}
  \item 神经网络的训练以及求导使用 C++ 实现,目前不支持多机训练,只支持 CPU
  \item 用 Golang 实现了简单的 Parameter Server,集成了 raft 进行容错
\end{itemize}
\end{onehalfspacing}

\datedsubsection{\textbf{Titan 索引优化}}{2014年10月 -- 2015年5月}
\role{Java}{毕业设计}
\begin{onehalfspacing}
对 Titan 的索引进行优化
\begin{itemize}
  \item Titan是基于 HBase 的开源图数据库,其原生索引只支持等值查询,有明显缺陷
  \item 在 Titan 原索引的基础上进行了改进,使其支持等值查询,并提升了其可用性
\end{itemize}
\end{onehalfspacing}

\section{\faUsers\ 实习}
\datedsubsection{\textbf{领英中国}}{2015年5月 -- 2015年9月}
\role{实习}{后端开发}
\begin{itemize}
  \item 基于开源的图数据库 Neo4j 搭建了位置索引,服务于线上的“附近的人”
  \item 对 mongo 进行了 archive, 按照时间查询对应的库或者表
\end{itemize}

\datedsubsection{\textbf{一点资讯}}{2016年3月 -- 2016年5月}
\role{实习}{算法工程师}
\begin{itemize}
  \item 对爬取的 APP 描述信息进行数据清洗以及无监督分类
  \item 对新闻进行 LDA 分解,对不同的评价指标进行判断。
\end{itemize}

% Reference Test
%\datedsubsection{\textbf{Paper Title\cite{zaharia2012resilient}}}{May. 2015}
%An xxx optimized for xxx\cite{verma2015large}
%\begin{itemize}
%  \item main contribution
%\end{itemize}

\section{\faCogs\ 个人能力}
% increase linespacing [parsep=0.5ex]
\begin{itemize}[parsep=0.5ex]
  \item 熟悉 C++,java,Shell 等编程语言,熟悉 linux 基本环境
  \item 熟悉常用的机器学习以及深度学习算法
  \item 了解分布式计算以及 TensorFlow 源代码
\end{itemize}

\section{\faTrophy\ 学术竞赛}
\datedline{\textit{银牌 (120th/3055)}~kaggle Allstate Claims Severity}{2016.09}
%% Reference
%\newpage
%\bibliographystyle{IEEETran}
%\bibliography{mycite}
\end{document}
