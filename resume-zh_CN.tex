% !TEX TS-program = xelatex
% !TEX encoding = UTF-8 Unicode
% !Mode:: "TeX:UTF-8"

\documentclass{resume}
\usepackage{zh_CN-Adobefonts_external} % Simplified Chinese Support using external fonts (./fonts/zh_CN-Adobe/)
%\usepackage{zh_CN-Adobefonts_internal} % Simplified Chinese Support using system fonts
\usepackage{linespacing_fix} % disable extra space before next section
\usepackage{cite}

\begin{document}
\pagenumbering{gobble} % suppress displaying page number

\name{何逸轩}
\centerline{求职意向: 分布式机器学习框架研发}
% \basicInfo{hyx\_ict@163.com}{(+86) 185-1023-8192}{北京市海淀区中国科学院南路6号}


\basicInfo{
  \email{hyx\_ict@163.com} \textperiodcentered\
  \phone{(+86) 185-1023-8192} \textperiodcentered\
  \linkedin[何逸轩]{https://www.linkedin.com/in/逸轩-何-b9156bbb/}}

\section{\faGraduationCap\  教育背景}
\datedsubsection{\textbf{中科院计算所}, 硕士}{2015.9 -- 2018.6}
\datedsubsection{\textbf{北京邮电大学}, 学士}{2011.9 -- 2015.6}

\section{\faUsers\ 项目经历}
\datedsubsection{\textbf{TensorFlow on Spark}}{2016年4月 -- 2016年12月}
\role{Java, C++, Scala}{}
\begin{onehalfspacing}
将 TensorFlow 移植到 Spark 上。整体使用 Java 开发,同时封装了部分 Tensorflow 的 C++ 接口
\begin{itemize}
  \item 支持同步、异步、单机训练、多机训练等多种训练方式
  \item 对 TensorFlow 进行了高层封装,支持 Layer 等高层接口
  \item 考虑 Spark 集群宕机的场景
\end{itemize}
\end{onehalfspacing}

\datedsubsection{\textbf{分布式深度学习框架}}{2016年6月 -- 至今}
\role{C++}{}
\begin{onehalfspacing}
自己实现的分布式深度学习框架,可以自配置神经网络,支持自动求导
\begin{itemize}
  \item 神经网络的训练以及求导使用 C++ 实现,目前不支持多机训练,只支持 CPU
  \item 实现了简单的 Parameter Server,集成了 Raft 进行容错
\end{itemize}
\end{onehalfspacing}

\datedsubsection{\textbf{Query To Query match}}{2014年10月 -- 2015年5月}
\role{Java}{}
\begin{onehalfspacing}
和微信的合作项目,预测2个 Query 的意图是否一致
\begin{itemize}
  \item 将 Query 转化为 矩阵,在此基础上利用 CNN 进行预测
  \item 设计了多种特征,最终线上的 accuracy 为 88.3\%
\end{itemize}
\end{onehalfspacing}

\section{\faUsers\ 实习}
\datedsubsection{\textbf{旷视科技}}{2017年6月 -- 至今}
旷视科技的内部自研深度学习平台 megbrain 的开发工作
\begin{itemize}
  \item 添加对 FPGA 设备的支持
\end{itemize}

\datedsubsection{\textbf{领英中国}}{2015年5月 -- 2015年8月}
\begin{itemize}
  \item 基于开源的图数据库 Neo4j 搭建了位置索引,服务于线上的“附近的人”
  \item 对 mongo 进行了 archive, 按照时间查询对应的库或者表
\end{itemize}

% Reference Test
%\datedsubsection{\textbf{Paper Title\cite{zaharia2012resilient}}}{May. 2015}
%An xxx optimized for xxx\cite{verma2015large}
%\begin{itemize}
%  \item main contribution
%\end{itemize}

\section{\faCogs\ 个人能力}
% increase linespacing [parsep=0.5ex]
\begin{itemize}[parsep=0.5ex]
  \item 熟悉 C++,Java,Shell 等编程语言,熟悉 linux 基本环境
  \item 熟悉常用的机器学习以及深度学习算法
  \item 了解分布式计算以及 TensorFlow 源代码
\end{itemize}

\section{\faTrophy\ 学术竞赛}
\datedline{\textit{银牌 (120th/3055)}~kaggle Allstate Claims Severity}{2016.09}
%% Reference
%\newpage
%\bibliographystyle{IEEETran}
%\bibliography{mycite}
\end{document}
